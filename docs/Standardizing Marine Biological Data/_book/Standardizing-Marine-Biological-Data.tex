\PassOptionsToPackage{unicode=true}{hyperref} % options for packages loaded elsewhere
\PassOptionsToPackage{hyphens}{url}
%
\documentclass[]{book}
\usepackage{lmodern}
\usepackage{amssymb,amsmath}
\usepackage{ifxetex,ifluatex}
\usepackage{fixltx2e} % provides \textsubscript
\ifnum 0\ifxetex 1\fi\ifluatex 1\fi=0 % if pdftex
  \usepackage[T1]{fontenc}
  \usepackage[utf8]{inputenc}
  \usepackage{textcomp} % provides euro and other symbols
\else % if luatex or xelatex
  \usepackage{unicode-math}
  \defaultfontfeatures{Ligatures=TeX,Scale=MatchLowercase}
\fi
% use upquote if available, for straight quotes in verbatim environments
\IfFileExists{upquote.sty}{\usepackage{upquote}}{}
% use microtype if available
\IfFileExists{microtype.sty}{%
\usepackage[]{microtype}
\UseMicrotypeSet[protrusion]{basicmath} % disable protrusion for tt fonts
}{}
\IfFileExists{parskip.sty}{%
\usepackage{parskip}
}{% else
\setlength{\parindent}{0pt}
\setlength{\parskip}{6pt plus 2pt minus 1pt}
}
\usepackage{hyperref}
\hypersetup{
            pdftitle={Standardizing-Marine-Biological-Data},
            pdfauthor={Brett Johnson},
            pdfborder={0 0 0},
            breaklinks=true}
\urlstyle{same}  % don't use monospace font for urls
\usepackage{longtable,booktabs}
% Fix footnotes in tables (requires footnote package)
\IfFileExists{footnote.sty}{\usepackage{footnote}\makesavenoteenv{longtable}}{}
\usepackage{graphicx,grffile}
\makeatletter
\def\maxwidth{\ifdim\Gin@nat@width>\linewidth\linewidth\else\Gin@nat@width\fi}
\def\maxheight{\ifdim\Gin@nat@height>\textheight\textheight\else\Gin@nat@height\fi}
\makeatother
% Scale images if necessary, so that they will not overflow the page
% margins by default, and it is still possible to overwrite the defaults
% using explicit options in \includegraphics[width, height, ...]{}
\setkeys{Gin}{width=\maxwidth,height=\maxheight,keepaspectratio}
\setlength{\emergencystretch}{3em}  % prevent overfull lines
\providecommand{\tightlist}{%
  \setlength{\itemsep}{0pt}\setlength{\parskip}{0pt}}
\setcounter{secnumdepth}{5}
% Redefines (sub)paragraphs to behave more like sections
\ifx\paragraph\undefined\else
\let\oldparagraph\paragraph
\renewcommand{\paragraph}[1]{\oldparagraph{#1}\mbox{}}
\fi
\ifx\subparagraph\undefined\else
\let\oldsubparagraph\subparagraph
\renewcommand{\subparagraph}[1]{\oldsubparagraph{#1}\mbox{}}
\fi

% set default figure placement to htbp
\makeatletter
\def\fps@figure{htbp}
\makeatother

\usepackage{booktabs}
\usepackage[]{natbib}
\bibliographystyle{apalike}

\title{Standardizing-Marine-Biological-Data}
\author{Brett Johnson}
\date{2020-04-08}

\begin{document}
\maketitle

{
\setcounter{tocdepth}{1}
\tableofcontents
}
Biological data structures, definitions, measurements, and linkages are neccessarily as diverse as the systems they represent. This presents a real challenge when integrating data across biological research domains such as ecology, oceanography, fisheries, and climate sciences.

\hypertarget{intro}{%
\chapter{Introduction}\label{intro}}

This is about stacking the right standards for your desired ineroperability with other data types. For example, interopating fish biology measurements with climate level variables. There are a few links neccessary to make this possible. This will permit ecosystem based models.

\hypertarget{data-structures}{%
\section{Data Structures}\label{data-structures}}

The OBIS-ENV Darwin Core Archive Data Structure.

\href{\%22https://obis.org/manual/\%22}{OBIS manual}

\hypertarget{ontologies}{%
\section{Ontologies}\label{ontologies}}

An ontology is a classification system for establishing a hierarchically related set of concepts. Concepts are often terms from controlled vocabularies.

From Marine Metadata:

"Ontologies can include all of the following, but are not required to include them, depending on which perspective from above you adhere to:

Classes (general things, types of things)
Instances (individual things)
Relationships among things
Properties of things
Functions, processes, constraints, and rules relating to things"

Unified Modeling Language?

\hypertarget{controlled-vocabularies}{%
\section{Controlled Vocabularies}\label{controlled-vocabularies}}

There are a number of controlled vocabularies that are used to describe parameters commonly used in a research domain. This allows for greater interoperability of data sets.

\begin{itemize}
\item
  \href{\%22http://cfconventions.org/standard-names.html\%22}{Climate and Format (CF) Standard Names} are applied to sensors for application with OPeNDAP web service.
\item
  \href{\%22http://vocab.nerc.ac.uk/collection/L05/current/\%22}{Device categories} using the SeaDataNet device categories in NERC 2.0
\item
  \href{\%22http://vocab.nerc.ac.uk/collection/L22/current/\%22}{Device make/model using the SeaVoX Device Catalogue} in NERC 2.0,
\item
  \href{\%22http://vocab.nerc.ac.uk/collection/L06/current/\%22}{Platform categories using SeaVoX Platform Categories} in NERC 2.0
\item
  \href{\%22http://vocab.nerc.ac.uk/collection/C17/current/\%22}{Platform instances using the ICES Platform Codes} in NERC 2.0
\item
  \href{\%22http://vocab.nerc.ac.uk/collection/P06/current/\%22}{Unit of measure}
\item
  \href{\%22http://vocab.nerc.ac.uk/collection/P04/current/\%22}{GCMD Keywords}
\item
  \href{\%22http://vocab.nerc.ac.uk/collection/C19/current/\%22}{Geographic Domain/Features of Interest}
\end{itemize}

There are numberous ways to investigate which controlled vocabulary to use and this can be fairly overwhelming. For a simplified overview see \href{\%22http://seadatanet.maris2.nl/v_bodc_vocab_v2/vocab_relations.asp?lib=P08\%22}{here}.

Note: To describe a measurement or fact of a biological specimen that conforms to Darwin Core standards, it's neccessary to use the `Biological entity described elsewhere' method rather than taxon specific.

\hypertarget{collections}{%
\subsection{Collections}\label{collections}}

\hypertarget{oceanography}{%
\subsection{Oceanography}\label{oceanography}}

\href{http://www.bco-dmo.org/}{Biological and Chemcial Oceanography Data Management Office}

\href{https://mmisw.org/ont/\#/}{Marine metadata interoperability vocab resources}

\hypertarget{biology}{%
\subsection{Biology}\label{biology}}

\href{http://bioportal.bioontology.org/ontologies/ECSO}{BioPortal Ecosystem Ontology}

\hypertarget{nerc-search-interfaces}{%
\subsection{NERC Search Interfaces}\label{nerc-search-interfaces}}

\begin{itemize}
\item
  \href{http://seadatanet.maris2.nl/v_bodc_vocab_v2/welcome.asp}{SeaDatanet Common Vocab Search Interface:}
\item
  \href{https://www.seadatanet.org/Standards/Common-Vocabularies/}{SeaDataNet Common Vocabularies:}
\item
  \href{http://seadatanet.maris2.nl/v_bodc_vocab_v2/vocab_relations.asp?lib=P08}{SeaDataNet Vocab Library}
\end{itemize}

\hypertarget{geosciences}{%
\subsection{Geosciences}\label{geosciences}}

`UDUNITS' are more common in geosciences

\href{https://www.unidata.ucar.edu/software/udunits/}{UDUNITS}

\hypertarget{ecoenvo}{%
\subsection{Eco/EnvO}\label{ecoenvo}}

\href{\%22http://www.obofoundry.org/ontology/envo.html\%22}{Environment Ontology} including genomics.

\hypertarget{wild-cards}{%
\subsection{Wild Cards}\label{wild-cards}}

\href{\%22https://www.bodc.ac.uk/resources/vocabularies/vocabulary_builder/biomodel/\%22}{P01 Biological Entity Parameter Code Builder}

\hypertarget{technologies}{%
\section{Technologies}\label{technologies}}

\hypertarget{erddap}{%
\subsection{ERDDAP}\label{erddap}}

\href{\%22https://coastwatch.pfeg.noaa.gov/erddap/index.html\%22}{ERDDAP} provides `easier access to scientific data' by providing a consistent interface that aggregates many disparate data sources. It does this by providing translation services between many common file types for gridded arrarys (`net CDF' files) and tabular data (spreadsheets). Data access is also made easier because it unifies different types of data servers and access protocols. \href{\%22https://github.com/HakaiInstitute/erddap-basic\%22}{Here} is a basic erddap installation that walks you through how to load a data set.

\hypertarget{semantic-web-and-darwin-core}{%
\subsection{Semantic Web and Darwin Core}\label{semantic-web-and-darwin-core}}

\href{\%22http://www.semantic-web-journal.net/system/files/swj1093.pdf\%22}{Lessons learned from adapting the Darwin Core vocabulary standard for use in RDF}

\hypertarget{resource-description-framework}{%
\subsection{Resource Description Framework}\label{resource-description-framework}}

\href{\%22https://dwc.tdwg.org/rdf/\%22}{Darwin Core Resource Description Framework Guide}

\hypertarget{literature}{%
\chapter{Literature}\label{literature}}

\hypertarget{methods}{%
\chapter{Methods}\label{methods}}

We describe our methods in this chapter.

\hypertarget{applications}{%
\chapter{Applications}\label{applications}}

Some \emph{significant} applications are demonstrated in this chapter.

\hypertarget{example-one}{%
\section{Example one}\label{example-one}}

\hypertarget{example-two}{%
\section{Example two}\label{example-two}}

\hypertarget{final-words}{%
\chapter{Final Words}\label{final-words}}

We have finished a nice book.

\bibliography{book.bib,packages.bib}

\end{document}
